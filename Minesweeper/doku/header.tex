% Deutsche Spracheinstellungen & Silbentrennung
\usepackage{polyglossia}
\setmainlanguage{german}

% Korrekte Anführungszeichen und Zitate
\usepackage[autostyle=true]{csquotes}

% --- SEITENGEOMETRIE & LAYOUT ---
% Seitenränder laut Leitfaden, Tabelle 2 [cite: 402]
\usepackage[
  left=4cm,	%Linker Seitenrand (Bindung beachten!!!)
  right=2cm,
  top=3.5cm,
  bottom=2cm,
  %footskip=2cm,
  a4paper
]{geometry}

% Zeilenabstand auf 1,5-zeilig setzen [cite: 404]
\usepackage{setspace}
\onehalfspacing

% --- SCHRIFTARTEN ---
% Empfehlung: Eine "seriöse" Schriftart wie Times New Roman oder Libertinus verwenden [cite: 402]
% XeLaTeX wird benötigt
\usepackage{fontspec}
\setmainfont{Times New Roman} % Oder z.B. "Libertinus Serif"
%\setsansfont{Arial}           % Oder z.B. "Libertinus Sans"
%\setmonofont{Courier New}      % Oder z.B. "Libertinus Mono"

% --- KOPF- UND FUSSZEILE ---
% [cite: 405]
\usepackage[automark]{scrlayer-scrpage}
\pagestyle{scrheadings}
\clearpairofpagestyles
\ihead{\headmark} % Innere Kopfzeile: Kapitel
\ohead{}          % Äußere Kopfzeile: leer
\ofoot[\pagemark]{\pagemark} % Fußzeile: Seitenzahl
\ifoot{}

\setlength{\footskip}{1.0cm}

% --- MATHEMATIK & PHYSIK ---
\usepackage{amsmath}
\usepackage{amssymb}
\usepackage{siunitx} % Für korrekten Satz von Einheiten

% --- GRAFIKEN & TABELLEN ---
\usepackage{graphicx}
%\graphicspath{{images/}} % Bilder im Unterordner 'images' ablegen
\usepackage{caption}     % Für Bild- und Tabellenunterschriften
\usepackage{subcaption}
\usepackage{booktabs}    % Für schönere Tabellenlinien
\usepackage{tabularx}

% --- LITERATURVERZEICHNIS ---
% Harvard-Stil (Autor-Jahr) 
\usepackage[
    backend=biber,
    style=numeric,%authoryear-comp,
    sorting=nyt,
    citestyle=numeric,
    %citestyle=authoryear,
    maxcitenames=2,
    maxbibnames=99,
    giveninits=true,
    uniquename=init,
    doi=true,
    url=true,
    isbn=false
]{biblatex}
\addbibresource{bibliography.bib} % Name der .bib Datei

% --- ABKÜRZUNGSVERZEICHNIS ---
% 
\usepackage[acronym, toc, nonumberlist]{glossaries-extra}
\setabbreviationstyle[acronym]{long-short}
%\input{abbreviations.tex} % Lädt die definierten Abkürzungen
%\makeglossaries
%\makenoidxglossaries

% --- VERLINKUNGEN IM PDF ---
\usepackage{hyperref}


% --- SONSTIGES ---
\usepackage[german]{cleveref} % Für intelligente Verweise, z.B. "siehe Abbildung 2.1"
\usepackage{float}


\usepackage{scrhack}
\usepackage{lipsum} % Nur für Fülltext, kann später entfernt werden

\usepackage{scrlayer-scrpage}

\usepackage{todonotes}
\usepackage{siunitx}
\sisetup{locale = DE}
\usepackage[section]{placeins}

\usepackage{pgf}
\usepackage{adjustbox}

\providecommand{\mathdefault}[1]{#1}

\usepackage{pdfpages}
\usepackage{multirow}
\usepackage{array}
\usepackage{textcomp}
